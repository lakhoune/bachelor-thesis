\chapter{Introduction}
Community Information Systems (CIS) help Communities of Practice (CoPs) in structuring their work by offering tools to coordinate and evaluate different tasks within the community. Researchers are active members of the CoP and contribute to the success awareness with the help of so called CIS success models \cite{Klam10c}. Success modeling requires technical know-how, which hinders the participation of less experienced members in the development process.

Meanwhile, chatbots are offering an intuitive interface with which users can perform tasks with varying degree of complexity. Therefore, the use of chatbots could increase the success awareness of members, which increases the overall success awareness of the CoP.

\section{Motivation}
The rise of the world wide web, gave researchers the opportunity to collaborate more easily. Researchers from far ans wide are given a place to meet and exchange information. Such a community of researcher is referred to as online Community of Practice (CoP) \cite{Renz08}.

The success of CoPs relies on various internal and external factors, such as the active participation of its members, or the general interest in the field.

Success is a dynamic process, as various needs of a CoP are changing over time \cite{Renz08,GKJa08}.
This is why the CoP needs to be continiously evaluated in order to predict future challenges and opportunities.

Nonetheless, such evaluations are often time-consuming and require data-management experts.
Automating the evaluation process and providing an intuitive interface helps Communities of Practice (CoP) implement continious success evaluations into  existing CIS, which in return helps them find ways to improve the community \cite{Renz08}.

Current success modeling techniques are often provided as a complicated web-interface, which requires some technical know-how to use them.
Moreover, success modeling is a social task, in which members of the commmunity should be included, but current web interfaces are often not optimized for collaboration.

Furthermore the rise of smartphones creates a new context which has not been taken into consideration by classic web interfaces: mobility.
Thus, new interfaces are required, which cope with the mobility context and which provide simple and intuitive collaboration possibilities.

\section{Thesis Goals}

Social networks, and chats specifically, have been a popular tool for information exchange and social interaction inside a community. Using a chat bot inside a CIS provides an interface, which is easier to learn and seams more natural to users. In addition, chat platforms are optimized for smartphones, which means that they take mobility into consideration. Therefore, the use of chatbots might adress the aforementioned issues.

Accordingly, the goal is to create a social bot, which can be used to define and visualize success models for existing las2peer services. The social bot should provide an interface to communicate with the MobSOS Continious Community Analytics (CCA) system.

The bot should be deployed with the Social Bot Framework and end users should be able to add the bot to a chat platform, such as Slack. The bot should be able to be integrated in group conversations in order to allow members of a community to collaborate more easily.

Overall, the goal of this thesis is to design a service, which combines the MobSOS framework with the Social Bot Framework and to determine if such a service improves the commmunity in terms of collaboration and success awareness.



