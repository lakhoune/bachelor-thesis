\chapter{Introduction}
Community Information Systems (CIS) help Communities of Practice (CoPs) in structuring their work by offering tools to coordinate and evaluate different tasks within the community.

Meanwhile, chatbots are used to assist CoPs by offering an intuitive interface to users. Users can perform tasks with varying degree of complexity, through the intuitive use of a chat platform. This makes it easier for non-experienced users to be included in the development process.

\section{Motivation}
The rise of the world wide web, gave birth to the concept of online communities, which reflect real-world communities and thereby allows for social interactions on the web. It was now possible for users to meet, exchange information and contribute to an online commmunity \cite{Renz08}.

The success of online communities relies on various internal and external factors, such as the active participation of its members, or the general interest in the field.

Success is a dynamic process, as various needs of a CoP are changing over time \cite{Renz08,GKJa08}. This is why the community needs to be continiously evaluated in order to predict future developments and challenges. However such evaluations are often time-consuming and require data-management experts. Automating the evaluation process and providing an intuitive interface, helps small Communities of Practice (CoP) implement continious success evaluations into their existing systems, which in return helps them find ways to improve the community \cite{Renz08}.

Current success modeling techniques are often provided as a complicated web-interface, which requires some technical know-how to use them. Furthermore success modeling is a social task, in which members of the commmunity should be included, but current web interfaces are often not optimized for collaboration.

Furthermore the rise of smartphones creates a new context which has not been taken into consideration by classic web interfaces: mobility.
Therefore new interfaces are required, which cope with the mobility context and which provide an simple and intuitive collaboration possibilities.


\section{Thesis Goals}

Social networks, and chats in general have been a popular tool for information exchange inside a community. Using a bot inside such a system provides an interface, which is easier to learn and seams more natural to non-trained users.

The goal of this thesis is to create a social bot, which can be used to define and visualize success models for existing las2peer services. The social bot should provide an interface to communicate with the MobSOS Continious Community Analytics system.

The bot should be deployed with the Social Bot Framework and end users should be able to add the bot to a chat platform, such as Slack. The bot should be able to be integrated in group conversations in order to allow collaboration between different members of a community.

Overall, the goal is to design a service, which combines the MobSOS framework with the Social Bot Framework and to determine if such a service improves the commmunity in terms of collaboration and success awareness.

such that members of communities can easily be integrated in the development process of the success model, which in turn increases the overall success awareness of the community.

\blankpage
