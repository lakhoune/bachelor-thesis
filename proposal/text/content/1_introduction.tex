\chapter{Introduction}
Community Information Systems (CIS) help Communities of Practice (CoPs) in structuring their work. They help such communities by offering tools to coordinate and evaluate different tasks within the community.

On the other hand chatbots can be used to assist CoPs by offering an intuitive interface to users. Users can perform various tasks with varying degree of complexity, through the use of chat. This reduces the complexity of those tasks, which makes it easy to use even for non-experienced users.

\section{Motivation}
The success of online communities relies on various internal and external factors, such as the active participation of its members, or the general interest in the field.

In order to measure the success of a community, we need to continiously evaluate the community. However those evaluations are often time-consuming and require data management experts. Automating the evaluation process and providing an intuitive interface, helps Communities of Practice (CoP) implement continious success evaluations into their existing systems, which in return helps them find ways to improve the community.
To provide such an interface the use of chatbots seems a viable option.

\newpage

\section{Thesis Goals}

\blankpage
