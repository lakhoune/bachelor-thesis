\chapter{Introduction}

\section{Motivation}
We want to continiously evaluate our systems. However those evaluations are time-consuming and require data management experts. The question arises how we can automate the evaluation process and provide an easy to use interface, such that Communities of Practice (CoP) can easily implement this into existing systems.
To provide such an interface the use of chatbots seems a viable option.

\subsection{Social Bots}
Bots in general are an interface, which provides automated services to end users. The most common bots which are also deployed in social media are social bots. Those social bots interact with users through a social media account.

In order to interact with a social bot, a dedicated text-channel is required. Such a channel is provided by most social media platforms in the form of chat. Social bots, which use chats to interact with users are so called chat bots. We can distinguish between retrieval based chatbots and generic chatbots \cite{NLKl19}.

Retrieval chatbots listen to certain keywords in a text and run tasks that are defined for such keywords. Generic chatbots on the other hand use machine-learning techniques in order to make assumptions about the user's intentions. An advantage of the generic chatbots is that they provide a better user experience as users can interact more naturally with them than with retrieval based chatbots. However the classification of the intentions can sometimes be wrong (false positives), which can also lead to a bad user experience if the task, which is performed, does not match the desired outcome.

In order to reduce false positives, the model which chatbots use to classify intentions needs to be trained on a labelled dataset. Such a dataset, which is also commonly called a trainingset, consists pairs of input data and desired output data. The higher, the qualitity, diversity and quantity of the trainingset, the better the resulting model will be able to classify intentions.
Once such a classifier is trained, it can be used by the chatbot to determine the intentions in chat and thereby run specific tasks that match those intentions.

The resulting chatbot is user-friendly, because the user is not required to memorize commands in order to get the desired results. From such standpoint, it becomes clear why such chatbots are desirable as assistants. Furthermore due to this even non-trained users can use such chatbots.
\newpage

\section{Thesis Goals}

\blankpage
