\chapter{Concept}
\section{Scenario: Mensa Community}
We illustrate the software system by considering a community of frequent Mensa visitors. The students in this community share a common interests which is the food at the canteen. Students can rate meals inside a chat application, called Slack, is designed with the help of the Social Bot Framework, which is deployed with las2peer. Furhtermore, the students can query visualisations through the chat application. Those visualisations include current success metrics which were previously defined by the community. The metrics can be defined inside the MobSOS Continious Community system. On the other hand the staff at the Mensa can gain insight into how successfull their canteen is and thereby improve their work.

\subsection{Use Cases}

\subsubsection{Issuing a visualization request} The MobSOS Continious Community system contains a mediabase with all the reviews of meals, made by students. Those can be accessed by a GraphQL interface. A user wants to know how active the community is. He opens Slack and asks the bot for a visualization. The bot issues a query to the GraphQL interface. The resulting data is then displayed as a chart inside the chat.

\subsubsection{Adding a review} A user can rate a meal inside the chat. The user therefore tells the bot which meal he had. The bot will start a dialogue, by asking questions about the meal. Each question results in a datapoint for the review. The final review is then added to the Mediabase database.

\subsubsection{Adding a new database} Core members and moderators of the community can add a new source of information by adding a new database to the system.

\subsubsection{Visualizations of multiple databases} Multiple databases can exist in a system. A user request a visualization of a metric which represents the overall success of the community. The resulting visualization contains results from all databases.
