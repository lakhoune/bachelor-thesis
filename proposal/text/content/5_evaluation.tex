\chapter{Evaluation}
In this chapter, we want to find out if communities could benefit from the use of bots in order to increase success awareness of the community.
The evaluation consists of two parts.

In the first part, we want to find out how easy it is to deploy the Social Bot into an existing CIS. Therefore a bot similar to the one described in section \ref{sec:structure} should be created, using the modeling space of the SBF. Participants should then be asked to connect the bot to a chat platform and then deploy it using the Social Bot Manager Service. Participants should be asked to add a database to the MobSOS CCA.

In the second part, we want to see if the bot is helpfull for visualization of success metrics and for success modeling. Futhermore, we also want to find out, if the bot encourages collaboration inside the community. In order to measure the collaboration aspect, participants should be divided into groups of at least three. Each group should then be asked to perform visualization requests. Participants should also be asked to update the success model. Both of these tasks should be performed in the original MobSOS CCA frontend and in a group chat, which contains the bot.

After fulfilling tasks from both parts, participants are handed out a questionnaire. The questionnaire should cover questions about usability and intuitivness of the bot, but also how users felt about using the bot, especially in terms of success awareness.
Finally they will also be asked how the chat bot compares to more traditional success visualization systems, like the MobSOS CCA frontend.


