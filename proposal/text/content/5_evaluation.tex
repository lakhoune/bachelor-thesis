\chapter{Evaluation}\label{cha:eval}
In this chapter, we want to evaluate how communities are impacted by the use of chatbots.
Specifcally, we want to see wether chatbots are favorable to traditional web frontends in regards to the following questions
\begin{itemize}
    \item Does the use of chatbots increase the success awareness of the community?
    \item Does it increase collaboration between members?
    \item Does it improve the user experience in terms of mobility?
\end{itemize}

In order to evaluate the community, an artificial community, as described in \ref{cha:concept}, is created.
The members of this community, will be volonteers, who frequent the mensa more or less regularly.
Thus, the community will consist largely of students and staff members of the university.
The evaluation will be conducted in two phases.

In the first phase, we want to familiarize the participants with the chatbot and the mensa service.
The participants will join a Slack group in which the Social Bot is deployed.
Furthermore, there will be a group channel in which users are encouraged to exchange Mensa related topics. The Social Bot will be a part of the channel and can be used to query the canteen menu.
Participants will also be able to make food reviews by contacting the bot in private chat.
Participants will be provided a questionnaire, which covers usability questions of the service.

In the second phase, we want to find out how easy it is to deploy the Social Bot into an existing CIS.
Furthermore, we want to find out if the bot is helpfull for visualization of success metrics and if the bot encourages collaboration inside the community, especially for success modeling.
In order to measure the collaboration aspect, participants should be divided into groups of at least three.
Each group should then be asked to update the success model.
This task should be performed in the original MobSOS CCA frontend and in a group chat, which contains the bot.
Afterwards, participants are handed out a final questionnaire.
The questionnaire should cover questions about how users felt about using the bot, when compared to the MobSOS CCA frontend, especially in terms of success awareness.


