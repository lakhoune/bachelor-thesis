\chapter{Introduction}
Community Information Systems (CIS) help Communities of Practice (CoPs) in structuring their work by offering tools to coordinate and evaluate different tasks within the community.
A CoP consists of members with various degrees of knowledge and interest for the field.
Researchers form the core-, and most active, members of the CoP.
They contribute to the success awareness of the CoP with the help of so called CIS success models \cite{Klam10c}.
Success modelling requires technical know-how, which hinders the participation of less experienced members, like students or hobbyists, in the development process.

Meanwhile, chatbots are offering an intuitive interface which users can use to perform tasks with varying degree of complexity. 
Therefore, the use of chatbots could increase the success awareness non-core members, which increases the overall success awareness of the CoP.

\section{Motivation}
The rise of the World Wide Web, gave researchers the opportunity to collaborate more easily. Researchers from all over the world can meet and exchange information more easily. A community of researchers studying a particular domain is referred to as (online) Community of Practice (CoP) \cite{Renz08}.

The success of CoPs relies on various internal and external factors, such as the active participation of its members, or the general interest in the field.

Success evaluation is a dynamic process, as various needs of a CoP are changing over time \cite{Renz08,GKJa08}.
This is why the CoP needs to be continuously evaluated in order to predict future challenges and opportunities.

Nonetheless, such evaluations are often time-consuming. Automating the evaluation process and providing an intuitive interface helps Communities of Practice (CoP) implement continuous success evaluations into existing CIS, which in return helps them find ways to improve the community \cite{Renz08}.

Current success modelling techniques are often provided as a complicated web-interface, which requires technical know-how to use them.

On the other hand, success modelling is a social process, in which members of the community should be included. However, current web interfaces are often not optimized for collaboration.

Furthermore the rise of smartphones creates a new context which has not been taken into consideration by classic web interfaces: mobility.
Thus, new interfaces are required, which cope with the mobility context and which provide simple and intuitive collaboration possibilities.

\section{Thesis Goals}

Social networks, and chats specifically, are a popular tool for information exchange and social interaction inside a community.
Using a chatbot inside a CIS as an interface is easier to learn and seems more natural to non-experienced users.
In addition, chat platforms are optimized for smartphones, which means that they take mobility into consideration. Therefore, the use of chatbots might address the aforementioned issues.

Accordingly, we created a social bot, which can be used to define and visualize success models for existing CIS. The social bot provides an interface to communicate with the MobSOS Continuous Community Analytics (CCA) system.

The bot is modelled with the Social Bot Framework and deployed with the las2peer Social Bot Manager. Users can add the bot to a chat platform, that allows the use of bots, such as Slack or Telegram.
The bot can be integrated integrated in group conversations in order to allow members of a community to collaborate more easily.

Overall, the main focus of this thesis is to design a service, which combines the MobSOS framework with the Social Bot Framework and to determine if such a service improves the community in terms of collaboration and success awareness.