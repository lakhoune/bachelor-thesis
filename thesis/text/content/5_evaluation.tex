\chapter{Evaluation}\label{cha:eval}
In this chapter, we want to evaluate how communities are impacted by the use of chatbots.
Specifically, we want to see whether chatbots are favourable to traditional web frontends in regards to the following questions:
\begin{itemize}
    \item Does the use of chatbots increase the success awareness of the community?
    \item Does it increase collaboration between members?
    \item Does it improve the user experience in terms of mobility?
\end{itemize}

In order to evaluate the community, an artificial community, as described in \ref{cha:concept}, is created.  The community is using a Slack workspace called \emph{Mensa Community} to exchange information.

The members of this community, are participants , who frequent the canteen more or less regularly.
Thus, the community will consist largely of students and staff members of the university.

\section{Collecting the requirements of the community}
The first step was to find out what the requirements of the community were and what success factors are most important to them. Therefore a survey was created which asked participants to order success factors based on their relevance. 
Furthermore, users were asked to list additional factors.
A success model was then created based on the factors from the survey. The resulting success model can be seen in Listing \ref{lst:successModel}

\begin{lstlisting}[language=XML,caption=Success Model based on requirements, label=lst:successModel]
<SuccessModel name="mensa" 
service="i5.las2peer.services.mensaService.MensaService">
    <questionnaires/>
    <dimension name="System Quality">
        <factor name="Low error rate"/>
        <factor name="Efficiency"/>
        <factor name="Fast responses"/>
    </dimension>
    <dimension name="Information Quality">
        <factor name="Reliability"/>
        <factor name="Accuracy"/>
        <factor name="Relevance"/>
    </dimension>
    <dimension name="Use">
        <factor name="Contributions"/>
        <factor name="Getting the menu"/>
        <factor name="Frequency of use"/>
    </dimension>
    <dimension name="User Satisfaction">
        <factor name="Enjoyment of use"/>
        <factor name="Number of user complaints"/>
    </dimension>
    <dimension name="Individual Impact">
        <factor name="Time to complete a task"/>
        <factor name="Time savings"/>
    </dimension>
    <dimension name="Community Impact">
    </dimension>
</SuccessModel>
\end{lstlisting}

\section{Evaluation setup}
The mensabot was setup in the Slack workspace. Members of the workspace could contact the bot through a private channel or add it to a public channel.

The evaluation was planned to be conducted in two phases.
In the first phase, we wanted to familiarize the participants with the chatbot, by allowing them to use the bot for an extended period of approximately one month. 
We also wanted to evaluated the bot's performance for providing canteen related information.
Users would be asked to use the bot whenever they went to the canteen. The idea was to collect logs and reviews, which would later be used in success visualizations. 
Participants would then have been asked to fill out a questionnaire, which covers usability questions of the service.  
Unfortunately, due to the Corona pandemic, the canteens were closed for most of the time. Therefore we decided not to do this first phase, but integrate it in the final evaluation.

% % The participants will join a Slack group in which the Social Bot is deployed.
% Furthermore, there will be a group channel in which users are encouraged to exchange Mensa related topics. The Social Bot will be a part of the channel and can be used to query the canteen menu.
% Participants will also be able to make food reviews by contacting the bot in private chat.

In the final evaluation, we wanted to find out how the bot affects the success awareness of the community.
We wanted to find out if the bot is helpful for visualization of success metrics and if the bot encourages collaboration inside the community, especially for success modelling.
We extended the success model of the community with example measures for each factor. We defined measures for the measure catalog which reflected the success factors of the success model.
Participants were asked to join the Slack workspace and contact the bot privately. They were asked to share their screen to be assisted in the case of any issues. They could ask the bot at any time what its capabilities were by typing \emph{help}.
The participants were then asked to perform three main tasks.

In the first tasks, they were asked to get the menu for a local canteen and make a review for a dish on the menu. 

In the second task, users were asked to visualize different success measures from the success model.

In the third task they were asked to update the success model by adding a measure from the measure catalog to a factor of their choice. 

Afterwards, participants were handed out a final questionnaire.
The questionnaire covered questions about how users felt about using the bot, especially in terms of success awareness.

% In order to measure the collaboration aspect, participants should be divided into groups of at least three.
% Each group should then be asked to update the success model.
% This task should be performed in the original MobSOS CCA frontend and in a group chat, which contains the bot.