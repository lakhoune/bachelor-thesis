\chapter{Conclusion and Future Work}

\section{Summary}
In our work, we analyzed the impact of a chatbot on collaboration and success awareness in the community. We have developed a chatbot that provided community-related information and served as an interface for success modeling.  We have evaluated the chatbot in an existing community.

We found that the use of a chatbot increased the success awareness of the community. Users were able to inspect the visualizations without any further help.
We also found out that the bot increases contributions in the community. Contributions might further increase by other means like Gamification. 
We confirmed that chatbots are favorable for quick and small tasks. Nonetheless, we noted that complex tasks were less intuitive and that intent recognition is challenging for complex sentence structures.

Thus we conclude that chatbots can be a great tool to assist a community and increases the success awareness of non-experienced users.  


\section{Improvements and Future Work}
As mentioned in our evaluations, the bots' usability can be improved.
Users frequently need to select an item from a list of options. 
The list of options should be enumerated, and the user can provide a number to select the item. This has been done for some tasks and should be easy to implement for the remaining tasks.
The items in the list should be clickable. The user can select an option by tapping on it.
The user should also be able to type the desired item from the list directly into the chat. We could use text mining techniques to comparing the users' input to the items from the list and select the most fitting one.

The review process needs to be revised. Users could add reviews more quickly by describing which meal they had and how many stars the meal was worth in one sentence. Consequently, users could add reviews more quickly. The resulting review should be shown after it has been saved. Moreover, it should also be possible to allow users to update them in the future. Lastly, the bot could also provide reviews for certain dishes.

The mensabot could also use recommender systems to show recommendations for certain dishes. The bot could study the users' behavior for writing reviews. If the user tends to add reviews for a particular dish, the bot could alert the user if that dish is served again in the future.

We should extend the Social Bot Manager to allow bots and services to distinguish between group chats and direct chat. It should also be possible to restrict certain actions in group chats. For example, the mensabot could be added to a group chat to get the canteen menu, but only allow writing reviews in private chat.
It should also be possible to filter certain messages. The bot should only respond to messages containing a keyword, e.g., the mentioning by name (\emph{@mensabot}).

We should extend the success modeling service to generate visualizations from the success model and display them periodically in chat. The bot could periodically generate \emph{reports}. Those reports include all the visualizations from the success model. 

The MobSOSb Evaluation Center needs to be updated to use the new data format for measure definitions.

We need to fix the picture upload for the Mensa Service. We could use the las2peer FileService to store pictures in the network. The bot can then also be extended to support picture upload. 
We also need to set up a Mediabase and a crawler such that reviews can be collected from outside the las2peer network. 

Finally, to increase contributions in the community, we should use \emph{Gamification}. A possibility would be to combine the las2peer Gamification framework\footnote{\url{https://github.com/rwth-acis/Gamification-Framework}}.